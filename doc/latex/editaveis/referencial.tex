\chapter[Fundamentação Teórica]{Fundamentação Teórica}
\label{cha:fundamentacao-teorica}

\section{Histórico da Teoria dos Jogos}
\label{sec:historico-da-teoria-dos-jogos}

Pode-se dizer que a análise de jogos é praticada desde o séculco XVIII tendo
como evidência uma carta escrita por James Waldegrave ao analisar um jogo de
baralho chamado \emph{Le Her} \cite{Prague_severalmilestones}. No século
seguinte, Augustin Cournot fez uso da teoria dos jogos para estudos relacionados
à política \cite{cournot_1838}. Mais recentemente, em 1913, Ernst Zermelo
publica o primeiro teorema matemático da teoria dos jogos \cite{zermelo_1913}.

Dois grandes matemáticos que se interessaram na teoria dos jogos foram Émile
Borel e John von Neumann. Nas décadas de 1920 e 1930, Emile Borel publicou três
artigos \cite{borel_1921} \cite{borel_1924} \cite{borel_1927} e um livro
\cite{borel_1938} sobre jogos estratégicos, introduzindo uma noção abstrada
sobre jogo estratégico e estratégia mista. Em 1928, John von Neumann demonstrou
que todo jogo finito de soma zero\footnote{Um jogo soma zero é um jogo no qual a
vitória de um jogador implica na derrota do outro.} com duas pessoas possui uma
solução em estratégias mistas \cite{neumann_1928}. Em 1944, Neumann publicou um
trabalho, junto a Oscar Morgenstern \cite{neumann_1944}, e com isso, a teoria
dos jogos entrou na área da economia e matemática aplicada.

Outro matemático que contribuiu para a área foi John Forbes Nash Júnior, que
publicou quatro artigos importantes para teoria dos jogos não-cooperativos. Dois
destes artigos \cite{nash_1950} \cite{nash_1951} provando a existência de um
equilíbrio de estratégias mistas para jogos não-cooperativos, denominado
\textbf{equilíbrio de Nash}, que será explicado na seção
\ref{subsubsec:equilibrio-de-nash}. Nash recebeu o prêmio Nobel em 1994, junto
com John Harsanyi e Reinhard Selten, por suas contribuições para a teoria dos
jogos.

\section{Conceitos Fundamentais da Teoria dos Jogos}
\label{sec:conceitos-fundamentais-da-teoria-dos-jogos}

Esta seção introduz os conceitos fundamentais da teoria dos jogos, tais como definição de um jogo não cooperativo, formas de representá-lo e teoremas para encontrar soluções.

\subsection{Definição de Jogo Não Cooperativo}
\label{subsec:definicao-de-jogo-nao-cooperativo}

Tendo em mente a definição de um jogo como sendo uma atividade interativa e competitiva no qual os jogadores devem obedecer a um determinado conjunto de regras, um jogo não cooperativo, então, não permite nenhum tipo de acordo entre os jogadores e o ganho de cada jogador é determinado pelo conjunto de regras \cite{jones_1980}.


\subsection{Representação de um Jogo}
\label{subsec:representacao-de-um-jogo}

Há duas formas de representar um jogo de uma maneira que seja possível analisá-lo em seguida, a \textbf{forma extensiva} e a \textbf{forma normal}. A forma extensiva faz uso de uma estrutura de árvore, onde os nós representam estados do jogo e as arestas representam as jogadas possíveis a partir daquele estado. Dado um jogo $\Gamma$, não cooperativo com \emph{n} jogadores tem-se:




\subsection{Soluções de um jogo}
\label{sec:solucoes-de-um-jogo}

\subsubsection{Teorema Minimax}
\label{subsubsec:teorema-minimax}

\subsubsection{Equilíbrio de Nash}
\label{subsubsec:equilibrio-de-nash}
