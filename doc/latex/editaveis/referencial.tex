\chapter[Fundamentação Teórica]{Fundamentação Teórica}
\label{cha:fundamentacao-teorica}

\section{Histórico da Teoria dos Jogos}
\label{sec:historico-da-teoria-dos-jogos}

Pode-se dizer que a análise de jogos é praticada desde o séculco XVIII tendo
como evidência uma carta escrita por James Waldegrave ao analisar um jogo de
baralho chamado \emph{Le Her} \cite{Prague_severalmilestones}. No século
seguinte, Augustin Cournot fez uso da teoria dos jogos para estudos relacionados
à política \cite{cournot_1838}. Mais recentemente, em 1913, Ernst Zermelo
publica o primeiro teorema matemático da teoria dos jogos \cite{zermelo_1913}.

Dois grandes matemáticos que se interessaram na teoria dos jogos foram Émile
Borel e John von Neumann. Nas décadas de 1920 e 1930, Emile Borel publicou três
artigos \cite{borel_1921} \cite{borel_1924} \cite{borel_1927} e um livro
\cite{borel_1938} sobre jogos estratégicos, introduzindo uma noção abstrada
sobre jogo estratégico e estratégia mista. Em 1928, John von Neumann demonstrou
que todo jogo finito de soma zero\footnote{Um jogo soma zero é um jogo no qual a
vitória de um jogador implica na derrota do outro.} com duas pessoas possui uma
solução em estratégias mistas \cite{neumann_1928}. Em 1944, Neumann publicou um
trabalho, junto a Oscar Morgenstern \cite{neumann_1944}, e com isso, a teoria
dos jogos entrou na área da economia e matemática aplicada.

Outro matemático que contribuiu para a área foi John Forbes Nash Júnior, que
publicou quatro artigos importantes para teoria dos jogos não-cooperativos. Dois
destes artigos \cite{nash_1950} \cite{nash_1951} provando a existência de um
equilíbrio de estratégias mistas para jogos não-cooperativos, denominado
\textbf{equilíbrio de Nash}, que será explicado na seção
\ref{subsubsec:equilibrio-de-nash}. Nash recebeu o prêmio Nobel em 1994, junto
com John Harsanyi e Reinhard Selten, por suas contribuições para a teoria dos
jogos.

\section{Histórico da Inteligência Artifical aplicada à Jogos}

No final do século XIX, Von Kempelen inventou um dispositivo chamado \emph{Maelzel Chess Automaton}, capaz de jogar uma partida inteira de xadrez \cite{shannon_1950}. Porém, vários artigos foram publicados após a exibição de tal máquina, incluindo um trabalho por Edgard Allan Poe entitulado \emph{Maelzel's Chess Player} explicando que a máquina era operada por uma pessoa. Uma tentativa mais honesta foi feita em 1914 por Torres y Quevendo, que construiram um dispositivo que jogava um jogo final de xadrez (rei e torre contra rei) \cite{vigneron_1914}. O objetivo da máquina era forçar um cheque-mate no jogador.

Há anos programadores tem se interessado em inteligência artificial com uma perspectiva de programar um computador que consiga ganhar de um ser humano \cite{samuel_1959} \cite{samuel_1967}

\section{Conceitos Fundamentais da Teoria dos Jogos}
\label{sec:conceitos-fundamentais-da-teoria-dos-jogos}

Esta seção introduz os conceitos fundamentais da teoria dos jogos, tais como definição de um jogo, formas de representá-lo e teoremas para encontrar soluções.

\subsection{Definição de Jogo Não Cooperativo}
\label{subsec:definicao-de-jogo-nao-cooperativo}

Tendo em mente a definição de um jogo como sendo uma atividade interativa e competitiva no qual os jogadores devem obedecer a um determinado conjunto de regras, um jogo não cooperativo, então, não permite nenhum tipo de acordo entre os jogadores e o ganho de cada jogador é determinado pelo conjunto de regras \cite{jones_1980}.


 ., pode ou não ser simétrico\footnote{Um jogo é chamado simétrico quando as regras é a mesma para todos os jogadores.}


\newtheorem*{JFN}{Definição. Jogo de n-jogadores na Forma Normal}

\begin{JFN}
é definido como uma tupla (2n+1)
$$(J; S_1,...,S_n; u_1(s_1,...,s_n),...,u_n(s_1,...,s_n))$$
onde $n \geq 2$ é um número natural; $J = \{1,2,...,n\}$ é um dado conjunto finito chamado \textbf{conjunto de jogadores}, onde seus elementos são os jogadores; $\forall i\in \{1,2,...,n\}$, $S_i$ é um conjunto arbitrário chamado \textbf{conjunto de estratégias do jogador} $i$, e $$u_i: S_1 \times S_2 \times ...\times S_n$$
\end{JFN}

De acordo com \cite{sartini_IIbienaldasbm}, a teoria dos jogos pode ser vista como "a teoria dos modelos matemáticos que estuda a escolha de decisões ótimas sob condições de conflito". Para isso, é preciso definir um jogo de tal maneira que os conflitos sejam evidentes. Uma maneira de representar um jogo não cooperativo é fazer uso de sua forma normal \cite{jones_1980}, no qual é feito uma matriz com cada estratégia de cada jogador. Para cada estratégia do jogador \emph{linha} tem-se uma estratégia do jogador \emph{coluna}, e o conflito é representado nesta célula da matriz.

\subsection{Representação de um Jogo}
\label{subsec:representacao-de-um-jogo}

Para analisar um jogo é preciso primeiro representá-lo de uma maneira que seja possível analisá-lo em seguida. Forma extensiva \cite{jones_1980} é uma das maneiras de representar um jogo, fazendo uso de uma estrutura de árvore, os nós representam estados do jogo enquanto as arestas representam as jogadas possíveis a partir daquele estado. Cada caminho\footnote{Um caminho na árvore é um conjunto de arestas partindo da raíz até uma de suas folhas.} nessa árvore representa uma estra



\subsection{Soluções de um jogo}
\label{sec:solucoes-de-um-jogo}

\subsubsection{Teorema Minimax}
\label{subsubsec:teorema-minimax}

\subsubsection{Equilíbrio de Nash}
\label{subsubsec:equilibrio-de-nash}
