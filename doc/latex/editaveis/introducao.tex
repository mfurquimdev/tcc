\chapter[Introdução]{Introdução}
\label{introduc}

\section{Contextualização e Justificativa}

A \textbf{Teoria dos Jogos} é uma área de estudos derivada da matemática que, por alguns anos vem estudando o comportamento de indivíduos sob uma situação de conflito, como em jogos, balança de poder, leilões, e até mesmo evolução genética \cite{sartini_IIbienaldasbm}. Esta área possui duas frentes de estudo: (a) \emph{teoria econômica dos jogos}, o qual possui motivações predominante econômicas, e (b) \emph{teoria combinatória dos jogos}, que faz uso dos aspectos combinatórios de jogos de mesa e não permite elementos imprevisíveis.

\section{Objetivos Principais e Secundários}

O objetivo deste trabalho é, fazendo uso da \emph{teoria combinatória dos jogos}, encontrar um \emph{winning move}\footnote{\emph{Winning move} é um movimento que, a partir de uma determinada jogada, garantirá a vitória independente do resto do jogo}. Além do objetivo principal, este trabalho ainda possui dois objetivos secundários: (a) estabelecer uma heurística\footnote{Heurística é uma abordagem para solucionar um problema sem garantias de que o restulado é a solução ótima.} para se maximizar o ganho (\emph{payoff}), fazendo uso da \emph{teoria econômica dos jogos} e; (b) criar uma inteligência artificial com três níveis de dificuldade para jogar contra um jogador.

\section{Estrutura do Trabalho}

O restante deste trabalho está organizado da seguinte maneira: Na seção \ref{cha:fundamentacao-teorica} é narrado uma breve história da teoria dos jogos e seus conceitos fundamentais, além de conter explicação para os temas de análise de complexidade, análise combinatória e programação dinâmica, e explicação das regras do jogo \emph{Big Points}. A seção seguinte (\ref{cha:metodologia}) lista os equipamentos, \textit{softwares} e metodologia utilizados para o desenvolvimento do trabalho e, também, a maneira que a foi analisado o jogo. Os resultados, até o momento, são descritos na seção \ref{cha:resultados-parciais}, o cronograma de trabalho na seção \ref{cha:cronograma}, e as considerações finais na seção \ref{cha:consideracoes_finais}.
