\documentclass[dvipdfm, a4paper, 11pt]{article}
\usepackage[brazil]{babel}
\usepackage[utf8x]{inputenc}
\usepackage{amssymb, amsmath, pxfonts, MnSymbol}
\usepackage{textcomp}
\usepackage[pdftex]{graphicx}

\begin{document}
\title{\textit{Big Points}\\Uma Análise Baseada na Teoria dos Jogos}
\author{Mateus M. F. Mendonca}
\date{\today}
\maketitle
\section{Resumo}\label{resumo}

A área de Teoria dos Jogos estuda as melhores estratégias dos jogadores
em um determinado jogo. Aplicando suas teorias em um jogo de tabuleiro
eletrônico, este trabalho propõe analisar o jogo \emph{Big Points} a
partir de um determinado estado e, como resultado, identificar as
melhores heurísticas para os jogadores e uma possível inteligência
artificial.

\section{Introdução}\label{introduuxe7uxe3o}

\begin{itemize}
\itemsep1pt\parskip0pt\parsep0pt
\item
  História da teoria dos jogos
\item
  Definição de teoria dos jogos
\item
  Jogo a ser analisado
\item
  \emph{Regra do jogo}
\end{itemize}

\section{Materiais e Métodos}\label{materiais-e-muxe9todos}

\begin{itemize}
\itemsep1pt\parskip0pt\parsep0pt
\item
  O jogo eletrônico está sendo implementado
\item
  Foi feito uma análise combinatória inicial para descobrir a
  possibilidade de computar todas as possíveis jogadas de um ou de
  vários jogos. Chegou-se na conclusão que não era factível
\end{itemize}

\subsection{Análise combinatória}\label{anuxe1lise-combinatuxf3ria}

O jobo \emph{Big Points} possui cinco peões de cores distintas, pode ser
jogado de dois a cinco jogadores, e dos 55 discos totais, cinco são
brancos, cinco são pretos e nove de cada um das cinco cores restantes.
Os peões podem estar em cima de um disco ou em uma das cinco posições da
escada. Cada partida diferente pode ser representado pela quantidade de
jogadores e pela posição inicial dos discos, compondo o tabuleiro. Desta
forma temos que o número de jogos distintos é 10\^{}41 \textless{}
\#Jogos Distintos \textless{} 1042 ∑

\section{Resultados}\label{resultados}

?

\section{Discussão e Conclusões}\label{discussuxe3o-e-conclusuxf5es}

?
\end{document}
