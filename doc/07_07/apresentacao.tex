\section{Tema}

\begin{frame}
\frametitle{Tema}
\begin{block}{}
 \textit{Big Points}: Um Estudo de Teoria dos Jogos
\end{block}
\end{frame}

\section{Etapas}
\begin{frame}
\frametitle{Etapas}
\textbf{String:} Game Theory
\begin{itemize}
\item Pesquisa pela \textit{String}
\item Leitura do \textit{Abstract}
\begin{itemize}
    \item \textbf{Capes Periódicos} 10 Artigos significantes
    \item \textbf{Ebrary} 11 Artigos significantes
\end{itemize}
\end{itemize}
\end{frame}

\section{sucupira}
\begin{frame}
\frametitle{sucupira.capes.gov.br/}

\textbf{Evento de Classificação:} "Qualis 2013"\\
\textbf{String:} "Game Theory"\\
"1099-1859; International Journal of Mathematics, Game Theory, and Algebra (B5)"\\
.\\
\textbf{Evento de Classificação:} "Classificação de periódicos 2012"\\
\textbf{String:} "Game Theory"\\
"0020-7276; International Journal of Game Theory (Print) (B3)"
\end{frame}

\section{Melhor entendimento}
\subsection{Definição de Game Theory}
\begin{frame}[fragile]
\frametitle{Definição de \textit{Game Theory}}
    A teoria dos jogos é uma teoria matemática criada para se modelar fenômenos que podem ser observados quando dois ou mais \textit{agentes de decisão} interagem entre si.
\begin{lstlisting}
    Game = <Players, Strategy, Payoffs>
\end{lstlisting}
\begin{equation*}
\begin{split}
  P_{layers}\ &=\ \{p_1, p_2,...,p_n\}\\
  \forall p_i \in P_{layers},&\ \exists s_{ij} \in S_i\\
  S_{i}\ &=\ \{s_{i1},s_{i2},...,s_{ij}\},
\end{split}
\end{equation*}
\end{frame}

\subsection{Keywords}
\begin{frame}
\frametitle{\textit{Keywords}}
\begin{itemize}
 \item \textit{Game Theory}
 \item \textit{Minimax Theory}
 \item \textit{Winning Move}
\end{itemize}
\end{frame}

\section{String melhorada}
\begin{frame}
\frametitle{String melhorada}
\begin{itemize}
 \item "\textit{Game Theory Minimax Theory Two-person Zerosum Game}"
\end{itemize}

\begin{itemize}
    \item \textbf{EBSCOhost} 8 Artigos significantes
\end{itemize}

\end{frame}
